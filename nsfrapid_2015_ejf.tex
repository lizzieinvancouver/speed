\documentclass[10pt,letter,oneside]{article}
\renewcommand{\baselinestretch}{0.99}
% \renewcommand*{\thefootnote}{\fnsymbol{footnote}}
\usepackage{sectsty,setspace} 
\usepackage[top=1.00in, bottom=1.0in, left=1in, right=1in]{geometry} 
\usepackage{graphicx}
\usepackage{epstopdf}
\usepackage{amsmath,latexsym,amssymb,wasysym}
\usepackage{natbib}
\usepackage{textcomp}
\usepackage{gensymb}
\usepackage{wrapfig}
\usepackage{hyperref}
\usepackage{float}
\usepackage[font={small,it}]{caption}
\usepackage{multicol}
\setlength{\bibsep}{0.0pt} % squeeze bibliography
\usepackage[font=small,skip=0pt]{caption} % squeeze more space

\setlength\parindent{0pt} % no indents throughout
\setlength{\bibsep}{0.0pt}

\parskip=5pt
\pagenumbering{arabic}
\pagestyle{plain}
% squeeze space
\linespread{0.99}
\addtolength{\itemsep}{-0.05in}
 

\newenvironment{smitemize}{
\begin{itemize}
  \setlength{\itemsep}{1pt}
  \setlength{\parskip}{0pt}
  \setlength{\parsep}{0pt}}
{\end{itemize}
}

\usepackage{fancyhdr}
\pagestyle{fancy}
\fancyhead[LO]{RAPID proposal}
\fancyhead[RO]{Wolkovich}

\newcommand{\Section}[1]{\vspace{-8pt}\section{\hskip -1em.~~#1}\vspace{-3pt}} 

\begin{document}

One sentence: This research proposes to utilize the world's largest collection of grapevine accessions---before its impending shut-down and move---to study the correlation of phenology and other important plant functional traits.\\

% Title options
RAPID: Phenology as a
 functional trait: Utilizing the hyperdiversity of grapevines as a model system to understand complex functional relationships\\

Plant phenology---the timing of reoccurring life history events such including leafing, flowering and senescence---is the most reported biological impact of climate change with most estimates reporting shifts of 1-3 weeks in spring leafout over the past several decades. Further, growing research on phenology suggests it is likely an important biological indicator of other community and ecosystem shifts, with links to species invasions, local extinctions and the functioning of ecosystems, including carbon sequestration and nutrient cycling. Yet, despite its apparent importance to understanding and predicting the consequences of climate change on ecological systems, it has been most often studied in isolation. This has left plant phenology research generally outside of the greater functional trait literature and limited our understanding of whether phenology itself is a critical biological indicator or one of a suite of traits that co-vary with climate change. To date there is some evidence that important functional attributes such as plant growth rates and strategies co-vary with the timing of flowering and budburst, but a more mechanistic understanding of drivers of phenological variation requires the full integration of phenology into studies of plant function. \\

Here, we propose to study how phenology and phenological plasticity co-vary with a suite of major plant traits by exploiting the hyperdiversity of phenology and other functional traits found within winegrapes (\emph{Vitis vinifera} subsp \emph{vinifera}). Increasingly, researchers have been using winegrapes as a model system given their tremendous functional diversity, well-documented genetic diversity and vast collections across the globe. We propose to work at `Domaine de Vassal' (henceforth `Vassal'), a French agricultural research station, located in the coastal town of Marseillan, that has been a globally important resource of grapevine diversity for well over a century and, due to leasing issues and sea level rise, will move its entire collection within 18 months. Vassal houses 7,500 accessions of grapevines from 47 countries representing greater than 2,400 different grape varieties including wild species, mutants and cultivars. The winegrapes represented in this collection alone exhibit two months of diversity in flowering time, as well large variation in other traits such as leaf morphology, vein density, water use and heat stress responses. \\

Work using winegrapes as a model system to study functional traits has been hampered, however, by an agricultural reality: due to the \emph{Phylloxera} pest, which destroys roots of \emph{Vitis vinifera}, cultivated grapes must be grafted onto North American wild grape species. This reality results in researchers having to unscramble the dueling signals of traits expressed in the plant. A critical and globally-unique feature of the research site at Vassal, however, is its location on the beach sand soils which render resistance to phylloxera and thus negate the need to graft onto non-native rootstock. This presents a unique opportunity to study phenological and plant trait variation independent of rootstock-scion interactions. Further, robust phenological data at multiple stages (i.e., budburst, flowering, and veraison) has already been collected and recorded for the vast majority of the collection. \\

Unfortunately, the entire collection at Vassal is in the beginning stages of being moved inland to a site where all plants will have to be grafted onto new rootstock, a process that will take several years, making the use of this invaluable collection time-sensitive. We propose to collect a suite of plant functional traits related to growth strategies, water and nutrient use across phenological stages from a diverse set of winegrapes that span the entire range of phenological diversity present in the collection. We will digitize relevant phenological records and build phenological models from the data to best describe the phenology and phenological plasticity of each variety sampled. 

The results of this work will advance our general understanding of how phenology is integrated into a complex of plant traits relevant to forecasting  plant and community responses to global change, with cascading implications for predicting climatic responses of an economically important horticultural crop. This project will support the mentorship of one postdoctoral researcher, foster international collaboration, develop outreach for early career scientists, advance a new model system for ecology, all while providing invaluable information from a collection unique in the world before its imminent shutdown.\\ 

\end{document}
